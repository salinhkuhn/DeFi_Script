\documentclass{article}
\usepackage{graphicx} % Required for inserting images
\usepackage[a4paper,left=2.5cm,right=2.5cm,top=2cm,bottom=2cm]{geometry}
\usepackage{minted}
\usepackage{lipsum} % Package for generating dummy text

\usepackage{graphicx}
\usepackage{subcaption} % for subfigures
\usepackage{mdframed}



\setminted{fontsize=\small, baselinestretch=1}


\title{Decentralized Finance \\
  \large {A Summary}
\author{Sarah Kuhn}
\date{April 2024}
}

\begin{document}
\maketitle
\thispagestyle{empty} % Remove page number from the first page

\newpage
\pagenumbering{arabic} % Start page numbering from 1
\setcounter{page}{1} % Set page counter to 1

\section{The Financial Market: Basics}
Let's start with some finance basics to understand why financial markets are so crucial. Financial markets are like giant networks where people and companies can invest money, raise funds, and manage risks. They  determine the prices of things like stocks, bonds, currencies, and also some more exotic investments like ETF's. These markets are essential for our economy because they allow  businesses to grow, people to invest their savings, and governments to manage their budgets more easily.

\subsection{Financial Markets: Underlying assumptions} 

Before we dive deeper into financial markets and their impact on our economy, let's first fix the foundation and basic assumptions upon which we build our whole study on:  Financial markets are incredibly complex because they are influenced by the decisions of billions of individuals. To tackle this complexity, we simplify things by making certain assumptions. One base assumption we make is that people act rationally and secondly, that markets operate efficiently. These simplifications helps us and professional economists to create models and theories which explain what we see happening on the markets and even make predictions about future trends. The assumptions described in the next subsection can be seen as the foundational building blocks that allow us to even make statements about financial markets and financial systems in general.

\begin{itemize}
    \item \textbf{Rational Expectations}

Definition: Rational expectations suggest that individuals form expectations about the future by considering all the information available to them at the current time point. And that they use these expectations to guide their decision-making process. These expectations are called "rational" because they align with all the information known at the time.

In Financial Markets: In financial markets, rational expectations mean that investors use all the information they currently have, like past prices, economic data, and news, to predict future asset prices. This is crucial because it indicates that asset prices mirror the demand for an asset and thus the overall sentiment of investors towards it. Prices typically show the demand for a product. Moreover, it implies that people won't consistently make mistakes when forecasting the future, so they won't be constantly wrong in assessing investment risks, because they take into account all available information. This leads to the market actors to be considered as rational actors.

%not sure if its understandble what i mean efficient markets ?

\item \textbf{Efficient Financial Market Theory}

Definition: Efficient market theory suggests that financial markets quickly and accurately incorporate all available information into asset prices. So "asset prices represent the best possible estimates of the risk attached to them" (from lecture slide). Thus the risk from a financial market or just an investment can be inferred through mathematical analysis and will be naturally integrated in an assets price.
\end{itemize}

\subsection{Role of Financial Markets} 
Financial markets play a key role in our economy by fulfilling several functions:
\begin{itemize}
    \item \textbf{Raising Capital}:
    We can think of financial markets as platforms for fundraising. Companies and governments turn to these markets to get investments from individuals and institutions. They can do this by selling shares of their company or issuing bonds to raise money. This "flow" of capital helps them fund their operations and start financing  new projects. So financial markets allow actors to raise capital.
    %can we say that in english ?
    
    \item \textbf{Economic signaling}: We could also call this "setting fair prices": In financial markets, prices of assets like stocks and bonds are determined by how many people want to buy or sell them. It is often refereed to as the demand and supply. So in financial markets the market force determines the asset prices which then helps investors know if they're currently paying a reasonable price for their investments.
    
    \item \textbf{Ease of trading}: The financial market provides a convenient platform for individuals to trade assets such as stocks and bonds whenever they need it. This flexibility is essential as it enables investors to quickly adjust their portfolios in reaction to changing conditions. It is often stated that financial markets offer liquidity to investors, as they ensure easy access to funds when needed.
    
    \item \textbf{Risk management}: In the financial market, individuals have access to tools to "secure" their investments from unexpected events that could lead to losses, such as sudden market fluctuations or in general unexpected bad events. This risk strategies/tools are summarized as "hedging against risk." By using instruments like options, futures, or swaps, investors can protect themselves from  movements that may affect the value of their assets negatively. Essentially, this involves investors gambling on the possibility of asset value loss. If this scenario occurs, meaning an asset losses worth and the investor hedge against it, the investor then profits as he has accurately predicted the loss. Another way of risk management is diversification, which involves spreading investments across different asset classes and categories (e.g., geographical or industry-based) to reduce overall risk. This strategy works as a loss incurred in one investment can be compensated by gains in others, which leads to a more balanced investing approach.
    
     \item \textbf{Economic growth}: Financial markets play a major role in fostering economic growth by providing capital to various economic actors such as companies and governments. This provision  of capital enables these entities to expand their operations and take on on new initiatives, such as investing in infrastructure projects. This can lead to the creation of more job opportunities and contribute to overall economic prosperity.
\end{itemize}

\subsection{Stakeholders in Financial Markets} 
In the previous section, we examined the role of financial markets within an economy. Now, let's dive deeper into the key participants, often referred to as stakeholders, in the world of Finance. A stakeholder stands for any individual or group capable of influencing or being influenced by the actions and consequences of financial markets. This "umbrella term" summarizes various actors such as:

\begin{itemize}
\item \textbf{Investors}:
Investors are individuals or organizations that supply funds to companies and governments in exchange for ownership(f.eg through buying a stock you are basically "owner" of a fraction of the company) or the promises of future returns. They come in various forms, from individual investors managing their personal portfolios to institutional investors like pension funds, which pool money from multiple investors and then invest it on their behalf. Investors typically acquire financial assets such as stocks, bonds, or also ETFs, aiming to grow their wealth over time through positive returns on their assets. You can also think of it like the investors lend their money to companies with the belief that in the future the company will have a return because they used the money profitable. And then as an investor you will get parts of this big win because you were so nice and provided liquidity through lending your money.

\item \textbf{Issuers}:
Issuers, as their name implies, are entities that issue financial instruments. They are organizations such as companies or governments, which provide financial products to investors to raise capital for diverse purposes. By offering financial instruments like stocks or bonds, issuers gain access to funds, thereby gain liquidity. They can utilize these funds for various projects, such as financing projects, expanding operations, or repaying their existing debts. For instance, when you purchase a company's stock as an investor, we refer to the company as the issuer of the stock.

\item \textbf{Intermediaries}:
Intermediaries act as "the man in the middle", bridging the gap between investors and issuers by making the trading of financial assets in the market easier. Think of them as the middlemen who coordinates the process of investing. Without intermediaries, investors would need to individually approach companies to discuss investment options, which wouldn't be  efficient nor scalabel. Hence, entities like banks and stock exchanges step in to provide a platform and information about services and the execution of the service itself. They are the main players in matching buyers with sellers and ensure correct transactions. In essence, intermediaries are vital for a correct and smooth functioning of financial markets. They massively contribute to market efficiency.

\item \textbf{Regulators}:
Regulators, as their name suggests, are tasked with overseeing and regulating the financial markets to ensure their fairness and stability and just overall integrity. Regulators establish and enforce rules, also known as regulations and standards, prescribing how interactions of market participants such as investors, issuers, and intermediaries should happen. The primary responsibility of regulators is to monitor market activities to maintain trust in the markets and guarantee the needed transparency. They also have an understanding of the connections and interdependencies among the financial institutions. Because of their overseeing position, regulators play a crucial role in preventing systemic risks that could lead to a collapse of the entire financial market, as has already occurred in the past f.eg in 2008. So they are like the big brother watching over the entire system.

\item \textbf{Market data providers}:
Market data providers are organizations responsible for collecting and processing information about financial markets. They gather data on asset prices, trading volumes etc and provide them to the market actors. Hence they offer valuable insights for investors,  issuers and even regulators. Many of our investment decisions are based on the data provided by these market data providers. We all agree that access to timely and especially accurate market data is key to make informed decisions on financial markets.
\end{itemize}

\section{Risk of Financial Markets}
Having explored the contributions of financial markets to the economy in the previous chapter, we now understand them as places where individuals engage in buying and selling various financial products, such as stocks and bonds. We can also say it is  the place where demand and supply meet. However, alongside the potential for profit, there also exists a considerable amount of risk. In this section, we will look at financial risks. With this we will also see where the notion of DeFi stems from and what the rise of DeFi promises.
\subsection{Market Vulnerability} 
Financial markets are exposed to a variety of different factors, each carrying its own unique risks.
\begin{itemize}
    \item \textbf{Systemic Risk}: Participating in financial markets naturally exposes the actors to systemic risk, which impacts the entire system as a whole rather than an individually actor. These risks come from various sources, such as political events or regulatory changes, and have an effects across the entire financial system.
    \item \textbf{Market Risk}:  These losses exist due to changes occurring directly within the market, such as fluctuations in exchange rates or declines in interest rates.
     \item \textbf{Credit Risk}:  Credit risks refer the the potential loss when the borrowers creditworthiness changed and he might be in financial distress and thus not able to repay the credit.
    \item \textbf{Operational Risk}:  This risk describes the potential of internal operations and intermediaries failing. An example would be an error in the transaction system or a fraud.
    \item \textbf{Cyber Security Risk}: This is a rather "new" and upcoming risk. As financial markets use more and more technology, cyber attacks on financial institutions increase and lead to loss of sensitive data, theft and other disruptions. 
\end{itemize}


\subsection{The DeFi promise } 
In the previous section, we explored how traditional hence centralized financial markets, where all transactions are routed through a central authority, have various risk potentials. Particularly after the 2008 economic crisis, trust in such centralized financial structures decreased. In November 2008, Satoshi Nakamoto released a software that started Bitcoin. He promoted it as a "fully peer-to-peer system with no trusted third party." For many, this marked the birth of DeFi systems and for many symbolizes freedom, privacy and anarchy. But what exactly are the key promises of DeFi systems?

\subsubsection{From CeFi to DeFi}

\begin{itemize}
    \item \textbf{Inclusivity}: Decentralized Finance aims to offer financial services to individuals all over the world, including those who lack access to traditional banking services.
    \item \textbf{Innovation}: DeFi platforms are mostly open-source and allow developers to create and deploy financial applications without needing approval from centralized institutions. This collaborative environment may encourage the creation of completely new applications and should push innovation.
    \item \textbf{Reduced Cost}: In DeFi, intermediaries are often replaced by automated processes, such as smart contracts which we will encounter later on. As a consequence DeFi has the potential to offer lower costs compared to traditional financial services as the third party in the middle isn't needed anymore.
    \item \textbf{Transparency}: Transparency is a, if not the key concept in DeFi. Technologies like the blockchain offer transparency by allowing every user to view and trace each transaction and actively engage wherever in the system. This transparency enhances or at least aims to enhance trust in the financial system for many individuals.
    \item \textbf{Financial Empowerment}: This is another key promise of DeFi: The promise to empower individuals by guaranteeing them greater control and access over their own assets. In DeFi, users often possess a private key, enabling them to directly engage in financial exchanges and transactions without needing any traditional intermediaries. But with this comes also handling its own risk.
\end{itemize}

\section{Extra: Smart Contract} 
This section will introduce smart contracts. While they weren't really covered in the course yet, I believe it's important to provide an introduction now, as the concept of programmable money/automation is fundamental to DeFi. All in all, smart contracts enable the automation of financial processes without the involvement of third parties, such as in centralized finance (CeFi) systems.\\ \\
Smart contracts are  self-executing contracts with conditions and rules directly encoded into their code. They are applications deployed directly on a blockchain, executing automatically when predefined conditions coded within them are fulfilled. Once deployed, a smart contract's code cannot be altered, which leads to a certain inflexibility but with it also a high level of trust for the transaction parties. By eliminating intermediaries, smart contracts stand for autonomy, which is a very important word in DeFi and also reduces the risk of fraud or manipulation. And as we eliminate a third party in the middle coordinating the transactions smart contracts allow us to reduce both time and cost which are usually associated with traditional contract executions.\\
\\
So, what exactly is the content of a smart contract? A smart contract serves as an agreement between two parties and handles transactions between them. These parties establish specific rules and encode them within the smart contract. When their defined conditions are met, the contract is fulfilled and it autonomously executes the transaction without requiring intermediaries like banks to issue them.

\subsubsection{Example in Solidity}%could improve this
Here, I will show some basic components of a smart contract. Various programming languages are utilized for writing smart contracts, but the most used language for the Ethereum blockchain which we focus on in the course is Solidity. It's a statically-typed language with syntax like JavaScript.
%go through code statements 

\begin{minted}[linenos, xleftmargin=20pt]{solidity}
// SPDX-License-Identifier: MIT
pragma solidity ^0.8.0;

contract SimpleExampleTransaction {
    address public owner;

    // Constructor to set the owner of the contract
    constructor() {
        owner = msg.sender;
    }

    // Function to send ether from the contract to a specified receiver 
    function sendEther(address payable _recipient, uint256 _amount) public {
        require(msg.sender == owner, "Only owner can send ether");
        require(address(this).balance >= _amount, "Not enough balance in the contract");

        _recipient.transfer(_amount);
    }

    // Function to get the contract's balance
    function getBalance() public view returns (uint256) {
        return address(this).balance;
    }
}

\end{minted}
\section{Oracles}
This chapter is about Oracles and their importance for a blockchain network. Oracles tackle the question on how to write data into the blockchain and keep the data up to date. One could say a downside of the blockchain technology is that a blockchain is an isolated database. Per construction a blockchain cannot access real word events or just browse the internet to get the newest data, thus has no access to off-chain events. Hence another mechanism is needed that allows us to integrate relevant data onto the chain.\\\\
In the context of Ethereum an Oracle refers to a trusted source of external, off-chain information that f.eg smart contracts can use to make decisions and thus perform actions based on these real-world events. Previously we learned that smart contracts are automated contracts that will execute as soon as their encoded condition is satisfied. So often real-word data like price tables, sports score or financial market data is requested. At this point Oracles then serve as a bridge between the contract on chain and the external data: Hence they fetch external data from off-chain sources, verify its authenticity and then provide this data to smart contracts on the blockchain.\\\\If we wouldn't have those Oracles the range of DeFi applications would be way more limited. 
For some an Oracle node might still seem very abstract, so the next subsection will go into more depth about these kind of components and especially how they are designed. It can be skipped.

\subsection{Oracle Analogy}
To me Oracles first seemed abstract and I couldn't really imagine what the structure behind such a node looked like. To start with, we can look at an simple oracle analogy. Imagine you are locked in a room, studying while having access to all the textbooks and resources you need to complete your assignments (data stored on the blockchain). However, you're not allowed to leave the room or access any information from the outside world. Now, let's say you're working on a project that requires information about the temperature outside. But since you're stuck inside the room, you can't see what's happening outside. Now the oracle is like a trusted friend who has the ability to go outside the room (access external data sources) and check the data you need for you. So your friend goes outside, checks the weather, and comes back to tell you (provides external data to the smart contract). Now based on the info provided by your friend (Oracle), you make a decision about the future of your project (execute actions in the smart contract based on external data).
\\
\\
So all in all an Oracle node is a piece of software/service that runs on existing computing infrastructure and interacts with external data sources to retrieve and provide data to smart contracts on chain. We can kinda think of it as an application that is installed and executed on a computer/server. It is responsible for fetching the requested data from external sources like the web and then transmitting it to smart contracts on the chain. There exist multiple, actually many many, Oracle nodes and each runs on dedicated servers, virtual machines or even private computers. In decentralized systems we have so called Oracle networks, Oracle nodes distributed across multiple locations and operated by different participants. Collectively they then provide reliable Oracle data service to smart contracts. If one as an participant provides an Oracle, it mostly refers to providing the computational resource to run the piece of software, hence provide the oracle service. Anyone who wants to, can actually provide an Oracle service to the blockchain. This means deploying software or infrastructure that retrieves, verifies and then transmits data from the external world to smart contracts on a blockchain. 

\subsection{Oracle Design Challenge}
Now that we know what Oracles are and why the are relevant, the questions is how do we efficiently integrate Oracles in our DeFi system?\\
\\
First of all, when dealing with Oracles the reliability of the data they provide is crucial. If one Oracle provides incorrect data, it can potentially destroy the outcome of smart contracts relying on that data. To reduce this risk various approaches to guarantee accuracy of Oracle data are used. Some strategies are:\\
\\
\textbf{Multi-sourcing}: Instead of relying on a single Oracle, smart contracts can fetch data from multiple independent Oracles. By looking at and collecting data from multiple sources, we (i.ex the smart contracts) can afterwards use techniques such as calculating the median, the weighted averaging, or detect outliers to filter out potentially wrong and disturbing data points. In this way we reduce the risk of malicious data  and in general improve reliability of the data.\\
\\
\textbf{Majority Voting/ Consensus}: In a previous Parallel Programming course, we looked at consensus protocols, which are now relevant in the context of Oracles. Consensus protocols help prevent the integration of unreliable data from an Oracle into our blockchain. By comparing data provided by multiple Oracles, we can determine the majority, aka the consensus value. This refers to the value they agree on or have in common, which becomes the final data we work with. While the implementation of consensus can involve simply comparing values, it often requires more complex, so called consensus algorithms.\\\\Now that we have tackled the challenge of ensuring reliable data in our smart contracts, the next problem is how to manage fails? Whether the fails occur in an Oracle node or within the source from which we want to retrieve the real-world data, how do we handle them ?
\\\\These fail scenarios are called Oracle outages and/or source outages and refer to situations where the Oracle service or the original data sources providing information to smart contracts become unavailable. How we handles these is important as f.eg an smart contract execution gets delayed or has an incorrect outcome because of missing or wrong data. Nowadays there are some used strategies to react upon such outages:\\
\\
A possible strategy is to implement \textbf{redundancy}. This means using multiple independent Oracles or data sources, so if one becomes unavailable, the smart contract can switch to another reliable source without interruption. So even when an Oracle outage occurs the other Oracles can take over and thus ensure continuity and reliability of the data.
The same idea holds for an source outage. Instead of having one external source with the relevant information, we fetch it from different websites, thus reducing the impact of an outage of a single website. Another possible approach is to implement default values into the smart contract or make them use historical/previously used data. This strategy is called the \textbf{"Fallback Procedure"} and is an effective way to handle outages. As the name suggest we "fallback" to old data values.\\
\\
As we see there are a lot of questions and also options to consider when planning on how to design and implement an Oracle network, further challenges the reader may think about are:

%\insert question from the slides still but was to lazy at some point bc didtn intrest me hahah

\subsection{On-Chain Oracles}
An additionally design possibility or rather functionality we will look at are on-chain Oracles. It might sound contradictory as Oracles are mostly used to retrieve external off-chain data but obviously one can also make use of them to determine the price of on-chain assets. Intuitively it makes sense as there most likely also exists smart contracts that need data from the blockchain f.eg the current market price of a certain cryptocurrency. On-chain Oracles are the ones providing that information to the smart contracts. So on-chain Oracles are a type of Oracle service where the data retrieval and transmission occur entirely within the blockchain. They don't interact with external data sources like traditional Oracles but retrieve data directly from within the chain. As mentioned the fetched data may include information stored in other smart contracts, blockchain transactions or other "state variables" from within the blockchain network. On-chain Oracles can then also pre-process the data before transmitting them to the smart contracts. f. eg the can aggregate data from multiple sources, apply maths formulas or even algorithms. Finally they then transmit the gained data directly to smart contracts within the same (!) blockchain network. This transmission occurs through blockchain transactions or function calls between smart contracts. On-chain Oracles are especially useful for getting prices for DeFi applications, where the smart contracts require real-time access to prices f.eg as in liquidations which we will see later. But one has to keep in mind that on-chain Oracles also bring certain risks as we will see in the security chapter. They can be manipulated and thus destroy the functionality of smart contracts.

\section{Censorship}

In this chapter we will look at the possibility of censorship on different layers of a blockchain and the implications it has. Censorship can be and is implemented at different layers of the blockchain. To look at that we will first recap the different layering's also seen in the lecture and asses where censoring of transaction could happen.

\subsection{Blockchain layers}
A blockchain can be imagined similar to a computer network with its different layers. Each layer responsible for different aspects of the blockchains functionality. Layers often referred to when we talk about blockchains are:\\
\\
\textbf{Network Layer}:The network layer stands for the communication and data propagation among nodes in the blockchain. It has the famous peer-to-peer communication protocols for transmitting transactions, blocks, and other network messages between nodes.\\
\\
\textbf{Consensus Layer}: The consensus layer mainly is responsible for the rules and protocols for achieving agreement among nodes on the validity and ordering of transactions in the blockchain. Different consensus mechanisms are implemented at this layer to handle how nodes reach consensus (agree on certain services and data) and produce new blocks.\\
\\
\textbf{Block Production}: This layer manages the process of block creation and propagation within the blockchain network. It selects the block producers and appends new blocks to the blockchain according to the consensus rules.\\\\
\textbf{Smart Contract Layer}: The smart contract layer enables the execution of the "programmable agreements". As we have seen, smart contracts are self-executing contracts with predefined conditions and actions encoded. The stand for the automation of transactions without needing intermediaries.\\
\\
\textbf{(Decentralized) Application Layer}: The application layer contains decentralized applications built on top of the blockchain.
\\
Like with computer networks organizing the blockchain in distinct layers brings modality and each layer contributes to the overall functionality and security of the blockchain network.


\subsection{Censorship Types }
To talk about censorship we first have a look at the definition of censorship. Censorship is know as the phenomena of suppressing or restricting certain data to be published or transmitted. In the context of DeFi it is refereed to as the manipulation or even complete surpression of financial information like transactions. It is often claimed that DeFi, especially the blockchain avoids having censorship. But however there are scenarios where censorship can still occur in various forms within the DeFi system. We distinguish between weak and strict censorship. Obviously there is no strict line defining weak and strong censorship but here are some examples which might provide some intuition.
\\
\\
\textbf{Weak Censorship}: 
Imagine a specific mining pool selectively blocking or delaying a certain transaction. This means they repeatedly refuse to include certain transactions into a block to mine. The mining pool could decide to censor activities from certain addresses and never include the transaction stemming from there. In weak censorship there might still be some other mining pools that process the transaction and include it in their suggested block. So the transaction can still be execute at some point in time but with some delay and inconveniences for the user.\\
\\
\textbf{Strong Censorship}: 
Strong censorship involves more effort and coordination of the whole system to completely suppress a transaction or even modify the blockchain's transaction history f.eg. to invalidate certain transactions. This sometimes happens when certain issues arise like a smart contract getting exploited or in general security issues appear. The system then "rolls back" to a state before the "bad" event occurred. Now it seems as the event has never happened. But for that the whole system has to be informed and involved. We see that strong censorship really needs consensus and coordination among the whole system while weak censorship can happen more locally.\\\\
The question now is how can we as users avoid to be censored or how can we apply censorship in a positive way and actively censor malicious actors ?
\subsection{Censorship Resistance: Mixer}
A solution attempt on the Ethereum blockchain is \textbf{Tornado Cash} . Tornado Cash is a so called mixer which allows users to make transactions more private by breaking the link between the sender and receiver blockchain addresses and "encrypting" the transaction amount. We as users can deposit ETH into the Tornado Cash's smart contract ( actually a set of smart contracts) and as we deposit money there we can choose an anonymity set. This defines our desired level of privacy, the larger the anonymity set we choose, the more private our transaction will be ( 1/ cardinally of anonymity set). Additionally a mixer allows us to only place a certain amount of ETH f.eg 1 ETH  at a time because transaction amounts are also data channels to backtrack a transaction. After depositing, the Tornado Cash mixer will mix our transaction which means that it gets combined with other activities from the anonymity set. This the breaks the link between the sender and recipient address. Finally after the mixing we can withdraw our funds to a new address. Technically we could withdraw it from the same address but then the whole point of privacy and avoiding censoring is lost. All in all Tornado Cash and in general mixers make it more difficult for outside observers to track financial activities on the blockchain. This is useful if you want to  protect your financial activities from analysis of other parties. I still want to mention that the use of tornado  cash doesn't necessarily provide 1oo\% privacy f.eg if you have to interact with a DeFi App to reach the mixer etc.
\begin{figure}[h]
    \centering
    \includegraphics[width=0.5\textwidth]{tc.png} % Replace 'example-image' with the filename of your image
    \caption{Tornado Cash Interactions, \scriptsize{source: lecture 2}}
    \label{fig:image-example}
\end{figure}

\subsection{Security Implications Of Censorship}
Assume a node is censoring transactions, now the questions is what does that node have to perform ? If a node doesn't just censor all of the incoming transactions, it has to go through a predefined list containing all the black-listed addresses. Thus it has to loop through this list and perform computations, check whether to censor it or not, hence has to use computational power on that transaction even though it will be drop at the end. Now as an attack one could just generate a big batch of transactions an spam the node. For this to work, transaction creation must be cheaper than transaction validation. The node then has to process this transactions and check if it should execute them. So we see that on the one hand censorship's is important to provide some legal compliance and fight against malicious users but on the other hand it is a platform for denial of service attacks and slows down transactions.
\subsubsection{Extra: Denial of Service Attacks:}
In this subsection we will quickly take a closer look at denial of service attacks on censoring nodes. We know that a Ethereum transaction involves different actors from validators to miners etc. and each of those nodes can be responsible for censoring certain addresses. A DoS attack on those censoring nodes in a blockchain consist of flooding the network with transactions to overwhelm the nodes responsible for censoring transactions. So the nodes are occupied with other computations than censorship and thus leads to censorship policies getting ignored or increased congestion in the system.\\\\The following piece of code shows the structure of a computationally heavy transaction that could execute such a DoS attack:

\begin{figure}[h]
    \centering
    \includegraphics[width=0.5\textwidth]{dos-attack.png} 
    \caption{Example Code Of An Expensive Transaction, \scriptsize{source: lecture 2}}
    \label{fig:DoS-attack}
\end{figure}


\section{Decentralized Exchanges}
In this chapter we will look at so called DEX's, decentralized exchanges. In the financial world exchanges are a center piece of any financial market. When it comes to exchanges the offers cover a wide range of sophistication levels so the choice depends on the users  preferences. A characteristic of a DEX is that it operates directly peer-to-peer without having a single trusted authority in the middle. But how do this exchanges operate ? How are the decentralized exchanges implemented ?\\\\To make sure that we all are aware of why exchanges are useful and needed: Exchanges are platforms for individuals to buy, sell, and trade assets, in DEX's theses assets are mostly cryptocurrencies. In exchanges buyers and sellers get matched and overall exchanges make it easier for investors and companies to find a corresponding match.
\begin{figure}[h]
    \centering
    \includegraphics[width=0.5\textwidth]{Bildschirmfoto 2024-04-02 um 13.49.50.png} % Replace 'example-image' with the filename of your image
    \caption{Position of DEX, \scriptsize{source: lecture 4}}
    \label{fig:DoS-attack}
\end{figure}

What are steps that have to be done ?
\begin{itemize}
    \item {price discovery}
    \item {price matching}
    \item {actual transfer}
\end{itemize}


\subsection{Order books}
Order books are the core components of DeFi exchanges. They are an essential tool for traders to understand market liquidity and the amount of demand in the exchange. In an order book usually buy orders and sell orders are each listed on one side with corresponding information about the price and the quantity. The orders get placed by a so called market maker which gets the order request and puts them into the exchange system. An important characteristic of order books is the order depth. It stands for the level/amounts of buy and sell orders currently placed in the exchange and thus for the liquidity of the market. F.eg if the +2\% is CHF 2 million, the impact of a buy on the price is way less than when we have less depth f.eg if  the +2\% is only CHF 1000. So the order book depth is a signal of the price impact a buyer/seller will have and therefore a signal for the market liquidity. In order books there a two main types of orders a user can place:
\begin{itemize}
    \item {Market order}: As the name suggest it is an order where the users wants to buy or sell at the current market price. Thus these orders try to execute immediately at the currently best available price.
    \item {Limit Orders}: Limit orders try to buy or sell at a minimum/maximum of a specific price or better. If we issue a limit order, the order gets placed into the order book and will be realized if the market ever reaches the desired price level.\\
    \\
    Often an exchange is either a market order or a limited order book exchange. We refer to decentralized exchanges with limited order book mechanism as \textbf{LOB DEX's}.
\end{itemize}

\textbf{Advantages of LOB DEX'S}:
\begin{itemize}
    \item {No KYC/AML}: 
    \item {No fees for exchange}: 
    \item {No impermanent loss}: 
\end{itemize}
\textbf{Disadvantages}:
\begin{itemize}
    \item {Other fees }: 
    \item {Slow execution}: 
    \item {Not fully decentralized}:  
\end{itemize}
The last aspect of order books we will look at is the match making mechanism. Order books are like a matching engine that matches sellers and buyers with the best option for them in the order book. If there is no corresponding match, it just places it into the book and waits until it can be eventually executed at some future time point.\\
\\
A keyword I want to mention here is the bit-ask spread. We will also see this definition in the "Quiz content chapter" but to summarize: The bit-ask spread describes how far away two participants, a buyer and a seller are. The closer/ smaller the bit-ask spread, the more they match and to more satisfaction each actors gets from the exchange.\\
\\
Where does the order book run? Earlier order books ran on a single server but as we can imagine running them on just a single server a lot can go wrong: A server can go down or become unavailable which would imply that the whole exchange is shut down and no one can trade. In addition single servers are vulnerable to flooding attacks where the server gets overwhelmed with requests and disrupted, thus doesn't react anymore. Moreover, relying on a single server means placing trust in a single entity, which contradicts the principles of decentralized structures that aim to avoid such centralization's. So running order books on a single server is considered outdated. Nowadays, order books operate on distributed systems, same as the blockchain itself.

\section{Automated Market Maker}
An \textbf{automatic/automated market maker (AMM)} is essentially a smart contract that autonomously performs market making functions, eliminating the need for a third-party. These AMMs are like algorithms that automatically determine asset prices based on mathematical formulas. These formulas most often rely on market liquidity which means how much we have of each asset. In AMMs, users looking to exchange or trade assets contribute liquidity to a liquidity pool. Each pool typically contains reserves for specific currencies and represents a particular trading pair, such as ETH to DAI and reverse. The AMM utilizes a "constant product formula" to calculate the price of a single investment in the pool. Therefore, when investors want to trade one currency for another, they interact with f.eg a DeFi application that interfaces with the corresponding liquidity pool. The AMM then computes the asset/trading price based on the ratio or one can call it "imbalance" between the two currencies in the pool. Usually the less we have of an asset, the more expensive it is. Consequently, the price of an asset in the pool fluctuates depending on trading activity.
\\
\begin{figure}[h]
    \centering
    \includegraphics[width=0.5\textwidth]{Bildschirmfoto 2024-04-02 um 15.09.34.png} 
    \caption{Constant Product Formula, \scriptsize{source: lecture 4}}
    \label{fig:DoS-attack}
\end{figure}
\\\textbf{AMM Pro Over Order Book }: Instant Liquidity\\
Automated market makers (AMMs) offer instant liquidity to us users to trade without waiting for a match like with LOB DEXes. There is always a match available (because usually there is always money in the pool), ensuring trading at any time.\\\\
\textbf{AMM Con Over Order Book:} Continuous Price Changes\\
The constant matching leads to more price changes, in detail:  AMM's are more sensitive to the quantity of assets being traded and constantly adapt their execution prices accordingly. So the price curve of an AMM is smoother in comparison of an LOB Dex order book, which has more "jumps". \\\\Automated market makers or in short AMM's are algorithms/protocols that make decentralized exchanges (DEXes) simpler and less reliant on third parties and remove the need for an order book to do the match making. Some well-known examples in the DeFi space that I want to mention are UniSwap and SushiSwap. UniSwap is one of the pioneers in AMM's using smart contracts and liquidity pools. SushiSwap, a DEX based on the UniSwap source code, introduced some extra features on top of the UniSwap implementation to attract more users. Despite the decentralized nature of AMMs, some centralized elements still remain: For example, when users need to vote on important decisions for an AMM, there must be an organization coordinating these votes. We call these organizations "decentralized autonomous organizations", short DAOs.  They handle and coordinate decisions for various DeFi protocols, such as AMMs and token protocols like ERC-20 which we saw in the lecture. (see Quiz Chapter for definition) \\
\\
To end this AMM chapter, here is an example of a constant AMM. This may provide a better understanding of how these protocols operate. It's important to know that in AMMs, the tokens involved are typically fungible (see Quiz Chapter for definition). Otherwise, defining an exchange ratio for these currencies becomes way more complex as we can't really objectively compare the tokens.
\begin{figure}[h]
    \centering
    \includegraphics[width=0.5\textwidth]{Bildschirmfoto 2024-04-02 um 15.16.43.png} % Replace 'example-image' with the filename of your image
    \caption{constant AMM, \scriptsize{source: lecture 4}}
    \label{fig:DoS-attack}
\end{figure}
\begin{itemize}
    \item {State before: In the pool are 10 assets of x and 30 assets of the currency y, now if we want to get 10 coins of the assets y, how many coins do we need to provide to the pool?}: 
    \item {As k needs to remain constant, we need to look at the equation and see how the parameters change. As we want to get 10 assets of y, the y amount will drop from 30 to 20 in the pool. Thus the x amount has to increase from 10 to 300/ 20 = 15. Hence we need to put 15-10 = 5 X coins into the pool.}:  
    \item {So if we provide at least 5 coins of x, the smart contract is satisfied and we will get 10 coins of y.}
\item {What happens if we add more liquidity to the pool, hence increase k ? In terms of prices nothing changes, only the constant product k is higher.}
\end{itemize}What we observe with AMM's is that the more we buy from an asset, the more expensive each additional piece/asset of the exact same type becomes. This phenomenon is known as \textbf{expected slippage}, which refers to the expected increase or decrease in price based on the trading volume and available liquidity. In simpler terms, buyers may end up paying a higher price for their order than the market price due to a lack of liquidity in the pool. We can think of it as if the assets became more rare and limited and thus more expensive. The more liquidity a certain coin has, the less impact a single trade of that coin has on the price, the reverse also holds.\\
\\
Additionally, there is the concept of \textbf{unexpected slippage}, which occurs when an unexpected purchase of the same asset front runs our trade, hence impacts our price execution. This can result in either a better or unfortunately a worse execution price for us.\\
\\
Now, we can ask ourselves if it's possible to incur a significantly worse price than first stated in the AMM. No, because there is the option to set a maximum slippage that we are willing to accept. If the slippage exceeds this limit, the transaction will fail.\\ \\
\textbf{Advantages of AMM}:
\begin{itemize}
    \item {no order book maintenance}
    \item {simple implementation based on formula}

\end{itemize}
\textbf{Disadvantages of AMM}:
\begin{itemize}
    \item {Danger of impermanent loss }: 
    \item {Slippage (especially in low liquidity markets)}: 
    \item {sandwich attacks (see extra chapter) }:  
\end{itemize}

\subsection{Impermanent Loss} %to do, have to finish s^this still only notes from lecture yet 
The effect/ loss we encounter through arbitrage. Called impermanent loss, because it is only a loss if we realize it aka withdraw from the pool. Else don't encounter the loss. Phenomena in AMM but not in order books.


\subsection{Pegged and Stablecoin AMM} 
In the lecture we were introduced to different types of Stablecoins: 
\begin{itemize}
    \item {Reserve-based}:For each coin, money is being collaterized in a bank. An example is USDC, for every issued USDC token on the blockchain, there is an equivalent coin in a bank. \\\\
    Here I want to make a quick input about USDC and USDT:\\ \\USDC and USDT are one of the most popular representatives of Stablecoins but what is the difference between the two? USDT, called Tether, is a fiat-backed Stablecoin and is designed to maintain a stable value relative to the US dollar. The company Tether issues these USDT coins and promises that each USDT coin is backed 1:1 with a traditional coin held in a bank account. It is claimed that USDT is primarily backed by the US Dollar but there are also some other fiat currencies and assets involved. Lately there have been more and more users claiming that Tether is not transparent enough because USDT full "backing" has never been verified.\\ USDC, the so called USD Coin, has similar promises to USDT. Each USD Coin has a stable value relative to the US dollar, where each USDC token represents one US dollar which is held in reverse somewhere. USDC is used by a collaboration of companies and is often perceived as more transparent and thus trustworthy compared to USDT. USDC is also more regulated than USDT, as USDC regularly undergoes checks to ensure the 1:1 ratio with the US Dollar.

    \item {Collateral based}: Here the idea is to overcollaterize each token. An example for this Stablecoin type would be MakerDAO, i. ex DAI. To retrieve DAI coins, the user has to place an 150\% worth of another assets in a special type of smart contract (see chapter liquidations). Collateral based Stablecoins are fully on chain.
    \item {Algorthmic}: Only their existence was mentioned in the lecture. As an short example they provided the Ampleforth(AMPL) coin. (More details about Ampleforth in the coin chapter)
\end{itemize}Before we look at those in more detail, here a general definition of a Stablecoin: A Stablecoin is a cryptocurrency that maintains a stable ratio/value relative to another currency or just asset in general. The aim of Stablecoins, as the name suggest, is to stabilize the currency, hence avoid high volatility which f.eg Bitcoin or Ethereum have. So they try to have as little price fluctuations as possible, which is attractive for investors who want less risk i.ex want to avoid the volatility certain other crypto-assets have.
In combination with Stablecoins the word  \textbf{"pegged" }often appears. A pegged cryptocurrency refers to a coin whose value is bound aka "pegged" to the value of another asset. Most often this other asset is a fiat currency like the Swiss Franc or resources like gold or silver. In general we would peg a currency to reduce its price volatility. Often the aim of the exchange rate is 1-to-1. To refer to previous sections we can also go back to the notion of slippage with Stablecoins. Stablecoins exhibit a more "linear" swap curve, hence more straight swap curve. This means they have a reduced risk of unexpected high slippage, as they are less volatile and less sensitive to the trading volume, so they are really more stable.\\
\\
Here I want to quickly mention that \textbf{Stablecoins and pegged coins} aren't the same thing, both exist independently. While Stablecoins are often pegged to a specific value or asset, not all pegged coins are Stablecoins. Both concepts are very similar, as far as I know every Stabelcoin is pegged but a pegged coin doesn't have to be a Stabelcoin. Stablecoins are often pegged to a specific asset like gold or a fiat currency, so they are designed to maintain a \textbf{stable relative value} to another asset. On the other hand, pegged coins refer to cryptocurrencies that are tied or pegged to the value of another asset, but not all pegged coins necessarily aim to be stable. So some pegged coins may experience price fluctuations while Stablecoins are always pegged to maintain stability. Here we should also mention that its not always fully 100\% decentralized, as often  third part like a bank or a DeFi App is involved to execute the exchange from the Stablecoin/ pegged coin to the value its tied to f.eg from USDC to Dollar .\\
\\
\begin{itemize}
\item {DAI}: One well-known Stablecoin cryptocurrency is DAI as we have encountered in the lecture. DAI is pegged to the US Dollar and as it is a Stablecoin (see Stablecoin Chapter) it is less volatile like Ethereum and usually maintains a stable value. DAI is created and managed by MakerDAO, a decentralized autonomous organization built on the Ethereum blockchain. The organization is based on smart contracts and works with collateralized debt positions (CDPs, see next section). Users can place f.eg ETH as collateral in CDPs and generate DAI tokens based on the value of their collateral amount. This process is also known as minting DAI.
\begin{itemize}
\item \textbf{How is DAI maintained stable?}: DAI stays stable through various methods. It relies on overcollateralization and liquidation of undercollateralized CDPs. This means that to obtain DAI, more collateral must be provided than the amount of DAI received (usually 150\% collateral). Moreover, if the value of the collateral drops below a certain threshold, the CDP is automatically liquidated. For instance, if someone's collateral falls below 150\%, MakerDAO will automatically sell their ETH holdings with a 13\% penalty fee. These mechanisms ensure that DAI is always backed by secure assets. When users repay their borrowed DAI, the amount of DAI is burned since DAI the DAI isnt isn't backed by collateral anymore. I would say DAI is pretty popular as it allows investors to hedge against other, more volatile cryptocurrencies but still allows users to have some stability and enjoy the benefits of DeFi. F eg. if you know that the price of Bitcoin might decrease , you could convert some of your Bitcoin into DAI. Since DAI's value is pegged to the US dollar, it's less likely to experience large price fluctuations compared to Bitcoin. So even if Bitcoin falls, the value of the DAI holdings should remain relatively stable. So you hedged against bitcoins volatility with DAI. But lately DAI interest rates have increased massively, which makes it a more expensive option.
Just for the sake of completeness I want to add another "easy" option to keep a currency stable, hence create a Stablecoin: One could just follow another Stablecoin, which means just use a lot of the other stable, non volatile currency as collateral.
\end{itemize}
\begin{itemize}
\item {\textbf{CDP's}, collateralize debt positions}: A collaterized debt position can be seen as a lock-up contract. It is a smart contract used by DAO's to  allow users to generate currencies like DAI. Users can lock up their assets as collateral in such CDP's and based on how much worth of collateral they put, the users can generate up to a certain amount of f.eg DAI. The generated DAI can then be used for various purposes such as trading or repaying other debt positions. If a user wants to retrieve their locked-up collateral, the must repay the generated DAI including an interest rate. After that, the CDP is considered as closed.
\end{itemize}
\end{itemize}
Now the questions is, what happens if a Stablecoin \textbf{de-pegs}, which means on of the coins a Stablecoin is tied to, drops to zero or in general loses a lot of worth. In CeFi systems, a phenomena called "bank run" would be observable. In DeFi, instead of a bank run, we see arbitrageurs attempting to exploit price differences by selling the less valuable coin in a stable swap and acquiring the more valuable one. However, this activity can lead to major instabilizations of the pool, and investors who enter the pool last suffer and lose the most. \begin{figure}[h]
    \centering
    \includegraphics[width=0.5\textwidth]{Bildschirmfoto 2024-04-02 um 16.33.28.png} % Replace 'example-image' with the filename of your image
    \caption{Blacklisting pools, \scriptsize{source: lecture 4}}
    \label{fig:DoS-attack}
\end{figure} Additionally, in such scenarios, some pools become blacklisted to avoid the "bank runs" or rather "pool runs". As seen in the censorship chapter, blacklisting a pool effectively restricts any movement of values into and out of the censored address.


\subsection{Sandwich Attacks}
Sandwich attacks are attacks where the attackers influence the order of execution to exploit slippage and convexity of the price curve in AMM's. They make use of the order to manipulate the prices: First, the attacker watches the network for a large transaction that will impact the liquidity of a decentralized exchange (Dex). When this transaction is announced, the attacker quickly sends his own transaction to the exchange. The attackers transaction aims to change the price of the assets involved in the big transaction. His transaction will influence the price of the assets in the Dex order book. Once the attacker's transaction is fully executed, the assets price is altered, possibly affecting all the pending transaction that follow. The subsequent transaction, often from a "normal" user, gets executed at the changed price, with the attacker profiting from the altered prices. The name "sandwich" comes from the fact that the user's transaction is sandwiched between the attacker's two transactions, leading to unexpected and unfair price changes. All in all the attacker makes profit from the changes in price caused by its first and then second transaction with the users transaction in between.\\
\\
Sandwich attacks occur when attackers manipulate the order of transaction execution to take advantage of "slippage and convexity of the price curve" of Automated Market Makers (AMMs). The procedure of a Sandwich Attacker: Initially, the attacker monitors the network for a significant transaction that will impact the liquidity of a decentralized exchange (DEX). This means he watches the network for a very large transaction.  Upon spotting such a transaction, the attacker quickly initiates their own transaction to the exchange. This transaction will influence the price of the assets involved in the large transaction. (Think of order book depth and the AMM constant product formula) Consequently, the price of these assets in the Dex order book is affected by the attacker's transaction. Once the attacker's transaction is fully executed, the price of the assets changes, potentially impacting all subsequent pending transactions. The transaction subsequent to the attackers transaction, often from a "normal" user, gets executed at the changed price, with the attacker profiting from the altered prices. All in all the attacker makes profit from the changes in price caused by its first and then second transaction with the users transaction caught in between.\\ 
\\
Now, one might ask: How does the attacker even know that the victim's transaction might be executed and placed in a block on the blockchain soon? Since we are in a blockchain-based system like Ethereum, all transactions are broadcasted to the entire network. Every participant, including the malicious attacker, is informed about all transactions.\\
\\
To avoid such unfair attacks like the Sandwich Attack DEXes have different attempts to especially avoid these front-running transactions (the attackers transaction being placed prior to the users transaction). They tried to implement increasing transaction fees or in general tried to improve transaction processing such that there is no time to issue another transaction beforehand. There are also notions where the execution order gets fixed before the content of each transaction is revealed. Disabling the attacker to spot large transactions. In the next subsection some protection mechanism against Sandwich Attacks are presented:\\

\begin{figure}[htbp]
    \centering
    \begin{subfigure}[b]{0.4\textwidth}
        \centering
        \includegraphics[width=\textwidth]{Bildschirmfoto 2024-04-02 um 16.56.24.png}
        \caption{Sandwich Attack Mechanism}
        \label{fig:img1}
    \end{subfigure}
    \hfill
    \begin{subfigure}[b]{0.4\textwidth}
        \centering
        \includegraphics[width=\textwidth]{Bildschirmfoto 2024-04-02 um 16.56.55.png}
        \caption{Sandwich Attack Prevalence}
        \label{fig:img2}
    \end{subfigure}
    \label{fig:both}
\end{figure}

\subsubsection{Protection Against Sandwich Attacks}
\begin{itemize}
\item {Slippage protection}: One option is to directly encode into the transaction what maximum percentage of slippage we tolerate. If this slippage level is over-shooted the transaction will not be executed. This form of protection is simple to implement but is rather handling the symptoms of sandwich attacks rather than disabling  the attacks themselves.
\item {Third Party Ordering}: We could let a trusted third party order the transactions. This then should happen in a manner such that the third party isn't biased towards the attacker. To avoid a malicious third party, trust is the key and decentralization focuses on avoiding the need of trusted third parties. In general, letting a third party to be involved is an efficient approach but contradicts the concept of decentralization.
\item {Committee Ordering}: Very similar to third party ordering, but committee ordering involves a whole committee of trusted third parties. With some consensus mechanism they agree on how to order the transactions. In committee ordering the influence of a malicious third parties is little but we encounter problems if they are multiple. Again this approach is relatively fair and efficient but still reduces decentralization.
\item {Commit and Reveal}: This approach executes in two phases. First the committed transaction get encrypted and an ordering on them is imposed. Only after their execution order is fixed, the transaction contents get revealed. Hence an attacker doesn't have access to a transactions content/amount until the order is fixed. This is a quite secure option but come with high costs and increases latency.
\item {First come, First serve}:
The "First Come, First Serve" approach can be compared to standing in line at a store. Whoever gets there first gets the item first. For DEXes it means that transactions are handled in the order they are received. This helps to stop sandwich attacks as DEXs prioritize the execution of transactions based on their arrival time. Thus an attacker cannot squeeze in a malicious transaction, if the users transaction was issued first. If the users transaction weren't issued first, the attacker wouldn't know that the transaction is large enoguh to attack. So this approach makes it less likely that attackers can exploit the order of transaction execution and manipulate prices to finally profit from the market.
\end{itemize}

%guest speech content 

\section{Guest Speech Content}

\subsection{Lecture 4: DeFi leverage }

In the first half of the lecture the lecturer introduced the concept of flash loans, leverage and arbitrage. Then a presentation about DeFi leverage presented by BIS, the Bank for International Settlements , followed. In this chapter I will just quickly cover the key points and definitions. 
\begin{itemize}

\item {Volatile and Stable Coins}: As we have previously seen a Stablecoin is a coin that tries to  maintain its value by pegging its price to a stable assets like a fiat currency. Volatile coins as the name suggests are more volatile, hence have a lot of price fluctuations and are not stabilized by any other asset. Examples of rather volatile coins are Bitcoin and ETH. This can be seen if one looks at the magnitude of their prices changes in the last couple of months.%may insert bitcoin and eth 

\item {Leverage}: In very simple terms leverage describes using borrowed money to invest and to increase the returns on an investment by using debt. In German on often refers to leverage as some sort of "Hebelwirkung". Through leverage a trader can borrow funds and increase the size of his position beyond what he would have only using his own capital. With bigger positions price movements have a bigger impact and thus leverage allows borrowers to increase potential gains.


\item {Delaveraging, False Delaveraging }: Deleveraging refers to an borrower paying down his debt or at least parts of it. So deleveraging is a synonym to "reducing the amount of debt one owns". So the goal is to reduce its own overall debt burden.\\
\\
False Delavering refers to a situation where an actor deleverages even though the reduction of his debt is not healthy nor sustainable. We talk about false delaveraging when actors manipulate their financial situation to make it appear as though they are reducing debt when they aren't. False deleveraging creates the illusion of financial health of an actor and hides his actually debt level.
Possible ways to false deleverage is just hiding debt position or as we say in German "abschreiben", which means putting them on an off-balance sheet.

\item {Close factor}: Refers to the maximum proportion of the debt that can be repaid in a liquidation. The close factor is a metric to measure the liquidity of an asset. It is usually calculated based on the trading activity of that position.A high close factor suggests that the security is rather illiquid because it frequently goes without any trading activity. A small close factor indicates that the security is liquid, as it has a trading activity on several days within the choosen time period.

\item {Shorting and Longing: }: Shorting and longing are both actually trading strategies often observable on the stock market. They make use either of falling or increasing asset prices. They are based on different believes on how the assets price will evolve. Short selling (shorting) is known as borrowing a certain assets from a lender and selling it on the market to someone else, hoping the after you sold it, the price will fall. Then after the price has fallen the next step in shorting would be to buy the asset back, but at the new, lower price. Finally you would return it to the lender with a difference as profit. Base idea of shorting is gambling that the asset price will fall and make use of it.//
//
Longing is the more intuitive trading strategy where you bet on the assets prices going up. First one would invest in an asset where you expect that its price will increase over time. "Going long" means buying an assets and keeping it with the belief that its value will increase such that you will sell it at a future point in time. When you sell it later you you expect that its price will be higher to make a profit. Here I want to mention that shorting and longing are strategies on how investors choose to invest their money, so it refers to what investors think and expect but not what has to happen.

\item {Negative correlation}: Is a term that is also often used outside of the finance world and refers to two or more parameters moving in the opposite direction. In DeFi negative correlation can be a trading/ investment strategy to diversify its portfolio. If two assets are negatively correlated, the price of of one assets goes up while the price of the other asset has the tendency to go down. So an investor can use this to compensate potential loses and reduce his total risk. In the previous paragraph we saw shorting and longing : For example if an investor makes a short selling (decrease in asset value) , he may also consider another asset which has a negative correlation with it, hence an expected price increase. That way, if the first asset's price does indeed fall, the second asset's price is likely to rise, so he could compensate any potential loss from the short. All in all, in finance negative correlation is a tool to optimize the portfolio by better managing risks.


\item {Pooling collateral}: 
Pooling collateral refers to combining securities/collateral's (f.eg a house, other assets etc) into a single pool to back up a loan. The borrowers put their collateral into a shared pool, which is then used as a security for the entire group of investors. Out of CeFi perspective this is very unusual as a lender has no real guarantee about the security of his loan. In DeFi it happens that lender take the good securities out of the pool and trade it against the less secure ones. An option to execute such a trade is with a flash loan. I think this collateral pooling is interesting as it as point where DeFi is actually less transparent than CeFi.

\item {Fixed Spread Liquidation Strategy}:
Is a way how an investor can manage its risks. In this approach the trader sets a predefined price level at which he will automatically exit the market, to limit a potential loss and at least secure some profit. For this the trader fixes a price spread, therefore the name, between his entry price (price he bought the assets) and the liquidation price (price at which he decides to sell to minimize the loss). Selling the assets and therefore exiting the market is often refereed to as "closing the position".

\item {Bad debt}: We consider a position a bad debt if its financially profitable neither for the borrower nor the lending platform to close the position. Often bad debt is debt that is believed to be uncollectable, hence the lender doesn't expect to get his money back. Bad debt is similar to an outstanding payment to a company that will never be paid because the customer is in financial distress. For the lender the goal is to design his risk management in such a way that as little bad debt as possible occurs.
\end{itemize}
\subsubsection{Conclusion:} %maybe improve it  !!!!!!!!!!!!!!!!!!!!!!!!!!!!!!!!!!!!!!!!!!!!!!!
This conclusion is rather personal and just summarizes what I benefited from the BIS presentation: It was mentioned that a unique feature of DeFi is the pooling of collateral when borrowing a loan. Unlike in traditional lending, where collateral is specific to each borrower, DeFi platforms put together various collateral types and place them in a big pool. This means that borrowers may not know exactly which assets are backing their loans. However, beside a lender not knowing how exactly his loan is backed up, pooling also offers opportunities, such as the ability to use flash loans to swap good collateral and leave unsafe assets in the pool. During the speech it was also mentioned that DeFi introduces metrics similar to those in CeFi, such as shorting, longing, and the close factor.

\section{Liquidations}
In traditional finance, the process of liquidation can be slow and complex as the illustration bellow shows. However,in DeFi, liquidations happen much faster, thanks to the automation provided by smart contracts. In this chapter we will take a closer look at how DeFi liquidations are executed.\\To ensure that it is clear what a liquidation is, here a brief explanation:\\\\ In general a liquidation refers to closing a certain debt position. For that we sell off the collateral assets that are backing the loan. We liquidate a position when certain conditions are not met anymore, these conditions are related to the health of the loan and kind off reflects the borrowers "trustworthiness". We need liquidations to reduce the risk of total loss for the lender. So as soon as a debt position undershoots a certain "health threshold" the system liquidates the position to protect the lender. On a higher level liquidations help maintain stability and integrity of the DeFi protocols.\\

\begin{figure}[h]
    \centering
    \includegraphics[width=0.5\textwidth]{Bildschirmfoto 2024-04-06 um 23.43.54.png} 
    \caption{Liquidation In Traditional Finance, \scriptsize{source: lecture 5}}
    \label{fig:DoS-attack}
\end{figure}

    \subsubsection{Typical Steps Of A DeFi Liquidation}
    -Threshold: We actively monitor the health of a debt position and measure certain ratios like the health factor and other values and if needed, trigger liquidation f.eg when the health factor drops bellow a certain threshold.\\\\
    -Collect collateral and valuation: As soon as the threshold is reached, we seize the collateral assets backing up the unhealthy debt position. The collected collateral is then valued to detect the current market worth.\\\\
    -Initiation/Selling: Now the DeFi protocol f.eg starts an auction to sell the collateral assets. Here I want to mention that there are several ways to sell the collateral, in the next section we will study the some of the most common approaches.\\\\
    -Debt Repayment: The money from the asset sale is now used to repay the outstanding debt associated with the liquidated debt position to the lender and any extra profit made will be distributed according to the protocols rules. (F.eg can be given back to the borrower or as an incentive for liquidators) \\\\
    -Position closure: After the debt is fully repaid and any extras are also distributed, the liquidated position is officially closed and the borrowers collateral is no longer held as the back-up for the debt.  Overall we protected the lender from losses due to illiquid borrowers and avoided a 100\% loss for the lender.


\subsection{Liquidation Types}
We quickly looked at the liquidations steps for a DeFi debt position. To sell the seized assets there are several approaches used by different protocols. We will introduce some of those:
\begin{itemize}

\item \textbf{Fixed Spread Liquidations}:%redo unhappy
Fixed spread liquidations in DeFi describe a process where a liquidator repays the debt of a borrowing position and acquires the collateral at a discounted price. The liquidator repays the debt and can obtain the collateral at a reduced price, which acts as an incentive for even participating in the liquidation process. The fixed spread represents the discount that the liquidator receives. When a position is liquidated, liquidators buy these assets the discounted price, hoping to sell them later at a higher price once the market stabilizes. One of the advantages of fixed spread liquidations is that they can be completed in a single transaction. However, to accurately determine the discounted price of the collateral, on-chain price oracles are required to monitor changes in asset prices. These oracles provide real-time data on asset prices, ensuring that the liquidation occurs at fair market price. Both on-chain and off-chain price oracles can be utilized for this aspect of fixed spread liquidations.
\begin{figure}[h]
    \centering
    \includegraphics[width=0.5\textwidth]{Bildschirmfoto 2024-04-07 um 17.47.44.png} 
    \caption{Fixed Spread Liquidations \scriptsize{source: lecture 6}}
    \label{fig:DoS-attack}
\end{figure}


\item \textbf{Auction Liquidations}:
In auction liquidations within DeFi, liquidators bid over a period of time  until the auction closes. This process typically involves multiple transactions on the blockchain. For instance, in the case of MakerDAO, auction liquidations were commonly used. The liquidation process relies on market forces, hence supply and demand, to determine the price of the collateral being auctioned. As more liquidators participate in the auction, competition increases, potentially resulting in a narrower spread ( => because of smaller discount ) for the liquidator. Auction liquidations often take longer to complete compared to fixed spread liquidations, which are completed in a single transaction. Additionally auction liquidations may incur higher transaction fees due to the increased number of blockchain interactions involved. So it can happen that the platforms incentives for liquidators get eaten away by high gas and transaction fees. Despite these drawbacks, auction liquidations allow for a fair market-driven price discovery process, which is considered beneficial for both borrowers and liquidators in the DeFi system.

\item \textbf{Flash Loan Based Liquidations}: So what if a user wants to participate in an liquidation but isn't in a financial position to repay the whole debt. Here flash loan liquidations come into play, which are only possible with liquidations strategies that happen in a single transaction. Hence they are useful for fixed spread liquidations or liquidations like in MakerDAO were they have options such as "tent" and "bent" liquidations. With flash loans liquidations, liquidators can borrow funds without collateral (flash loans) and use them to repay the debt of the borrowing position upfront. If the liquidation completes successfully within the same transaction, the flash loan is repaid along with any associated fees. The liquidator makes profit if the flash loan + fee is less than the market value of the bought collateral. Here I want to mention that flash loan based liquidations involve three types of fees, which reduce the liquidators spread: the fee for the flash loan itself, the fee for the liquidation transaction, and any additional transaction fees incurred on the blockchain. But despite the fees involved, flash loan based liquidations offer a convenient and efficient way for liquidators to participate in the DeFi system even if they don't have enough money to upfront repay.

\end{itemize}
\subsection{Important Terminology}
\textbf{Close Factor}:\\\\
The term "close factor" refers to a parameter used in decentralized finance platforms, particularly in the context of lending. The close factor determines the percentage of a borrower's debt that can be repaid during the liquidation process.\\\\In DeFi, users can borrow assets by collateralizing their holdings. But if the value of their collateral falls below a certain threshold, their position may become subject to liquidation. When this happens, the close factor determines how much of the borrower's debt the liquidators can pay off at once. We learned earlier the close factor is often calculated based on trading activity. Overall liquidations are attractive for liquidators as they can earn a profit if the market price of the collateral assets is higher than the discounted price at which the liquidator acquired them.\\ Thus often you want to buy as much of the collateral as possible. On many platforms the close factor is at 0.5, so a liquidator can repay up to 50\% of borrowers debt in one go. If they health factor drops bellow a defined threshold it can also be that the platforms let liquidators pay off 100\% of a borrowers debt.\\\\A helpful example I found online on medium.com is the following:\\
Assume "...the borrower deposits \$1,000 worth of \$ETH and the liquidation threshold is 80\%, then they can borrow up to \$800 worth of \$DAI and a liquidation can be executed when their collateral value drops below \$800.
If the collateral’s value drops below \$800, the liquidator can repay the \$400 loan to buy the collateral and pocket the bonus. Assume the  liquidation penalty is 5\% for \$ETH, so the liquidator effectively buys the borrower’s deposit for an equivalent price of \$1,000 / 1.05 = \$952.38."\\\\\textbf{Bad Debt}:\\\\Bad debt occurs when the value of collateral backing a loan falls significantly, making it undesirable for buyers. In the context of DeFi, bad debt may happen if the market value of the collateral decreases sharply, making it difficult to recover the outstanding debt through liquidation. This situation typically occurs when the collateral's price experiences a drastic decline.\\\\To mitigate the risk of bad debt, DeFi platforms may implement additional safety measures, such as reserve pools or insurance funds. These funds act as a buffer against losses resulting from bad debts. They provide liquidity to cover outstanding debts in case of collateral "loss". However, despite these safety measures, bad debt still occurs and remains a realistic possibility in volatile cryptocurrency markets.\\\\It's important for DeFi users to assess the risk of bad debt when lending. One should always use risk management strategies like diversification to minimize potential losses. Additionally, continuously monitoring collateral values helps to quickly react if the market changes and is crucial for minimizing the impact of bad debt.
\subsection{The Risk of Liquidations}

In the decentralized finance (DeFi) space, the risk of liquidations poses a significant concern for users as they want to avoid their position to get liquidated to quickly/early. \textbf{Market bot failure} is one scenario that can trigger liquidations. In the event of a market collapse or a sudden decrease in asset prices, panic selling may happen, causing gas fees (transaction fees) to rise. This increase in gas fees can create more liquidation opportunities, which leads to automated bots like some in MakerDAO to initiate liquidation transactions and liquidate certain user positions. However, there have been situations where these transactions failed to execute due to low gas fees. So it happend in the past that 8 million worth of collateral could be taken for 0 DAI.  All in all market bot failures can trigger liquidations and start a negative cycle resulting in even more liquidations. \\\\Moreover, a sudden increase in the amount of collateral available for liquidation can also further increase the risk. We know that liquidations present a lucrative opportunity for liquidators, it can also lead to a self-supporting negative cycle:  As liquidators sell off the acquired assets, the market has an increased supply of those assets. Thus accroding to demand and supply, the prices for those assets will be pushed down(similar to the constant product formula in AMM's). This can trigger additional liquidations for positions that are collateralized with these assets. So we result in a  \textbf{cycle of liquidations} and assets lose a lot of value. So we see that actively monitoring positions is important for risk management in DeFi systems and help to lower the potential of these negative cycles.

\subsection{Avoiding Liquidations}
Liquidations can be very beneficial for liquidators but as users/borrowers we want to avoid the liquidation of our position. For borrowers liquidations means losing their collateral assets or at least parts of it depending on the outstanding debt. Additionally a reputation damage can happen because of the undergoing liquidation. Everyone in the system knows you are being liquidated. This can negatively impact the accessibility to further loans. So as a user it is important to minimize the potential of a liquidation.\\
One tool to mitigate risk in decentralized finance are collateral swaps. These involve changing debt positions by swapping one type of collateral for another. For example, you might switch from using Ether as collateral to using USDC (a Stablecoin) instead. By executing this swap in a single transaction, you can maintain your position while reducing the risk of liquidation.\\\\But why would someone go for a collateral swap? One common scenario is when you expect a drop in the price of your current collateral asset and want to avoid liquidation. By swapping to a more stable asset, you can avoid liquidations and therefor any financial losses that come with it.\\
\begin{figure}[h]
    \centering
    \includegraphics[width=0.5\textwidth]{Bildschirmfoto 2024-04-07 um 17.48.13.png} 
    \caption{Collateral Swap \scriptsize{source: lecture 6}}
    \label{fig:DoS-attack}
\end{figure}

\subsection{Stability Measures}% to do it 

In DeFi , stability measures play a crucial role in assessing the "robustness" and reliability of DeFi protocols and debt positions. For example, they can help us avoiding or spotting position that need liquidation.\begin{figure}[h]
    \centering
    \includegraphics[width=0.5\textwidth]{Bildschirmfoto 2024-04-07 um 17.44.29.png} 
    \caption{What is "stable"? \scriptsize{source: lecture 6}}
    \label{fig:DoS-attack}
\end{figure}\\
One fundamental concept to understand is the distinction between exogenous and endogenous variables. Exogenous variables represent external  factors that are independent of the system itself being analyzed. For example, in MakerDAO, the price of assets such as Ether is considered exogenous because it is determined by the market and is not in control of MakerDAO .\\\\On the other hand, endogenous variables are internal to the system and are influenced by the actions inside the system. In DeFi we often look at endogenous variables to calculate stability measures f. eg. like the volatility. Another interesting metric is the  "worst price drop over a specific period". This gives us insights into potential downs of an asset. \\\\ Overall thanks to stability measures, we can analyze and better understand risks and opportunities that come with DeFi instruments. They are valuable tools for assessing the stability and resilience of DeFi platforms.\\

\section{Quiz Content}
Here is a collection of terms, questions and topics covered in the lecture quizzes. I tried to structure them according to topic.
\begin{itemize}
    
    \item \textbf{Financial Inclusion}: Refers to accessibility and availability of financial services to actors, especially in under-served communities. All parts of society should have access to a range of affordable financial services like bank accounts, investment opportunities and credit services. Financial education is an additional important aspect of financial inclusion. 
    (taken from lecture slides)
    
    \item \textbf{Collateral}:  Assets that a borrower pledges to a lender to secure a loan.
    If the borrow can't repay the loan, the lender will become owner of the collateral to recover the outstanding debt.
    
    
    \item \textbf{Over-collateralization}: Refers to pledging more assets than required by the collateral requirements. If the requirements says 150\%, the borrower would have to provide collateral worth 150\% of the loan amount. Any collateral valued more than those 150\% would refer to over-collaterizing.\\
    Benefits of over-collaterization is
    
    \begin{itemize}
     \item {Risk reduction}:
        Having more collateral provides an extra protection for the lender, hence reduces his risk in case of loss.  
    
    \item {Market volatility protection}:
    As with over-collaterization the lender has more protection also against market volatility's. Stock prices and in general the value of assets fluctuates, with a over collaterized loan, the lender has a buffer against these ups and downs.
     \end{itemize}

     
     \item \textbf{Market Makers}: :
        Market makers are individuals or entities that are like the backbone of a market. They make it easier for people to trade, set prices, and keep things running smoothly. They constantly buy and sell stocks, bonds etc. to publicly stated prices and in that way ensure a liquid market. They kinda ensure that there is always a counter party if a trader wants to transact.
    
    \item \textbf{Bid-Ask-Spread}: Difference between the buying (bid) and the selling prices (ask). The goal is to buy at low prices and sell for a high prices. The difference is then called the spread and is a compensation for having provided liquidity to the market.
        
    \item \textbf{Transparency}: Transparency is a big keyword in DeFi. Technologies such as blockchain are called transparent as they allow each user to see and trace every transaction in the system. For some this transparency property enhances trust for the DeFi system.

    
     \item \textbf{Financial Empowerment}: DeFi promises to empower the individuals by giving them more control and access over their assets. In DeFi the user often have a so called private key which allows them to directly engage in financial exchanges etc, without needing a traditional intermediaries.
    
    \item \textbf{Value at Risk aka "VaR"}: Is a statistical measure and describes the potential loss of an investment monitored over a time interval. VaR refer to an investment and a certain confidence level and estimate the maximum loss an investment is likely to incur with this confidence level.\\
    For example, a 95\% VaR of \$1 million over one week means that there is a 95\% probability that the portfolio will not lose more than \$1 million over the next week.

    \item \textbf{Expected shortfall aka CVaR}:
    Also known as Conditional Value at Risk (CVaR), is a risk measure that stands for the average loss that would occur beyond the VaR threshold. It represents the expected value of losses exceeding the VaR level.  

    \item \textbf{ERC-20}: ERC20 is a token-management contract for Ethereum-based tokens. This means it's a set of rules and guidelines i.ex like an interface for tokens. The characteristics of such ERC-20 tokens are that they are fungible aka every token is identical to every other ERC-20 token, which is a useful characteristics in DeFi applications. This is of importance because they can be exchanged in a 1-to-1 ratio. ERC-20 has become the most used standard for tokens on the Ethereum blockchain.
    (Btw ERC stands for "Ethereum Request For Comments" and mirrors how protocols and standards are discussed and developed within the Ethereum community. Each users can review and take part in discussion about these suggested standards before they actually get implemented). 

    \item \textbf{ERC-721}: As ERC-20 is a used standard for fungiable tokens, ERC-721 is a standard predominantly used for Non-fungible tokens (NFTs) in Ethereum. It represents a set of rules and guidelines for representing ownership of NFT's. Each token created using the ERC-721 standard is distinct from every other token, hence they are non-fungible as each token has unique properties. This implies that they aren't exchanged on a 1-to-1 basis.

    \item \textbf{Miners}: Miners are essential actors in the Ethereum network. They are responsible for including transactions into a block and creating new blocks. First Miners include pending transactions into blocks. Whenever a user wants to do a transaction (f. eg sending ETH to interact with a smart contract), it goes into the network and  miners will pick it up to process. The decision which miners can create the new block, the different miners compete to solve a puzzle. This process is computationally intensive. When a miner has successfully solved the challenge he can create and provide the new block. For their efforts miners get newly created ETH (for each successfully mined block). Additionally they can also have to transaction fees within the block they mined.

    \item \textbf{Transaction fees}:
    These are fees paid by users to miners to prioritize their transactions and ensure rapid of their transaction by miners.

    \item \textbf{Block generation}: It describes the process of creating new blocks and adding them to the blockchain. It is done by the miners in a blockchain network. First there has to be a collection of transactions waiting to be executed aka waiting to be in a block. Then the mining starts which means that miners compete in solving a puzzle, the first miner to solve it can create the new block and add it to the chain.

    \item \textbf {Liability}:  In DeFi liability refers to the responsibility that actors or whole institutions have in case something goes wrong with a DeFi item like a protocol or a whole platform. Unlike in traditional finance where it is more clear that liability is assigned to the centralized entities like banks, determining the liability in DeFi structures is way harder. As Ethereum operates on blockchain technology with smart contracts etc. there is no central authority to hold accountable, instead in DeFi it depends on different factors like protocol type, the actions of each users etc. Overall addressing liability in DeFi is less obvious and I think as DeFi continues to evolve, liability is one of the concerns that need to be tackled, especially if one wants ensure stability of these financial systems. In German liability refers to the term "Haftung".

    \item \textbf{(Fractional) Reserve}: This describes a banks reserve ratio. When we as customers deposit money into a bank, the bank is allowed to lend out a certain amount of that  money to borrowers or invest it into assets. The bank reserve ratio describes the fraction of the money that must be kept inside the bank, hence is not allowed be used to invest in other assets. Here I want to mention that adjusting a banks reserve ration directly influences the amount of money inside an economy. A lower ratio means banks have less money they can invest and lend to companies etc. Overall the bank reserve ratio is tool to control to manage the amount of circulating money and also ensures that banks always have enough when a customer wants to withdrawal money. it is calculated by: (money held by the bank) / (total money deposited by customers)
    
    \item \textbf{Anonymity Set}: In context of Tornado Cash, the Anonymity set is the amount of deposits in the mixer and refers to the level of privacy. The more transactions in the pool, the higher is the level of privacy. A Tornado Cash pool with 1000 deposits has a anonymity set size of 1000 and thus a privacy level of 1/1000.
    
    \item \textbf{Borrowing Capacity}: The borrowing capacity usually refers to the maximum amount of debt an actor is allowed to borrower from a lender. It is an important metric for lenders to asses the risk associated with a credit. A lot of different factors like income, existing debt and also employment influence the borrowing capacity. An interesting factor e look at in the course is the value of collateral. Lenders also look at the worth of collateral provided by the borrower. The more collateral the safer an credit seems.

    % work in progress, this is all still open.
 \item \textbf{USDC vs. USDT}:

\item \textbf{Under-Collateralization}: We have often heard of over-collaterziation for example in MakerDao. In contrast under-collateralization refers to the situation where the collateral has less value than the amount of money borrowed. I would claim this is a rather rare situation in DeFi but we have heard in the lecture that under-collaterized protocols exist often when they borrowed money is strongly tied to a certain purpose or condition, hence to borrower is limited in the use of the money.


\item \textbf{Base APY, Reward APY}: APY is the so called annual percentage yield (APY). It is an amount of money that the liquidator earns by providing liquidity to a DeFi protocol. It is called Base APY, as it is an interest rate that the participant receives per default without considering any additional rewards and incentives. The analogy for base APY in CeFi would be the interest rate we get simply for keeping our money in the banks savings account. We earn this base APY continuously over time based on the money deposited. For extra reward a reward APY is offered. The goal of the reward APY's is to encourage users to do certain activities like providing liquidity to liquidity pool or borrow assets. So the reward APY is like an extra that is given for those how do specific actions in the DeFi system. It is often that the reward APY is paid by the borrower to the lender. For better understanding here are some examples:\\\\
- Base APY: You deposit \$1,000 worth of stablecoins into a DeFi lending platform that offers a base APY of 5\%. This means that over the course of a year, you would earn \$50 in interest on your deposit.\\
- Reward APY: If we assume the platform also offers a reward APY of 3\% if a user provides liquidity to a specific pool. So, if you participate in the liquidity pool that offers the 3\% reward APY, you would earn an extra \$30 in interest over the year. 

 \item \textbf{Health Factor}: In DeFi the Health Factor is used to asses the health/ safety of a debt position. We usually compare the value of the debt and the corresponding collateral. It is calculated as follows:
\[ \text{Health Factor} = \frac{\text{Value of Collateral} \times \text{Liquidation Threshold}}{\text{Total Value of Debts}} \]
\begin{itemize}
    \item \textbf{Value of Collateral}: The total worth of the collateral, sum of the assets used as collateral by the user.
    \item \textbf{Liquidation Threshold}: A predefined threshold set by the lending platform. A position should never fall bellows this percentage else it will be liquidated.
    \item \textbf{Total Value of Debts}: The total value of the users outstanding debts.
\end{itemize}
Overall the Health Factor describes the ratio between the value of the collateral and the total debt owed. It describes how much of the total owed debt is covered by the minimum amount of collateral needed. If the Health Factor falls below a certain threshold, often set to 1, it signifies that the value of the collateral is not enough to cover the debts. So the users position is getting liquidated. So its a numeric representation of how close a position is to being liquidated. % watch a vidoe about it 

\item \textbf{Leverage}: We have encountered leverage already in previous chapters (see DeFi leverage) but to summarize: Leverage describes using borrowed money (additionally to its own equity) to invest and increase the potential return on an investment.

\item \textbf{Liquidation Spread}: By now we know that a "spread" usually refers to a difference between two values. The liquidation spread refers to the difference between the value of the collateral and the debt value at which a users position would become liquidated. We have learned that often users must maintain a certain level of collateral to avoid liquidation, and in this context the liquidation spread describes the percentage the collateral must exceed the debt value to keep the position. An example case would be a liquidation spread of 110\% which means the collateral exceeds the value of the debt by 10\%, if it were lower it could have been that the position was liquidated to repay the debt.\\
Here I want to add that the liquidation spread and the borrowing capacity are in a so called "target conflict" (in German it is the well-known term "Zielkonflikt").A high liquidation spread means more security for borrowers as it prevents them of being liquidated as soon as the collateral assets prices change a little bit but on the other hand it  also limits the users borrowing capacity. A lower liquidation spread allows for greater borrowing capacity but increases the risk of liquidation if asset prices decline.

Overall, the liquidation spread is an important parameter/ tool for risk management in DeFi protocols.

% to d o
\item \textbf{Arbitrage}: Arbitrage can be seen as finding a nice deal in trading. It is like buying something for a low price in one place and then sell it for a higher price somewhere else and make a profit in the process. In n simple terms arbitrage is about taking advantage of price differences between different markets or assets. We can look at an example to get better intuition:\\
Assume we notice that the price of Ethereum (ETH) is higher on one cyrpto exchange compared to another DEX. Now to take advantage of this price difference we buy ETH on exchange X (the cheaper one) and transfer the ETH to exchange Y (the higher priced one). At exchange Y we sell the ETH and by doing this we make a profit. Obviously we must consider transaction fees but exploiting the difference between these two exchanges is what is call basic arbitrage. 

\item \textbf{Liquidity Pools}: One can think of liquidity pools as big pools of cryptocurrencies or like big reservoirs. They are used in DEXes to enable and simplify trading. The content of liquidity pools are contributed by investors that want to make profit by providing their assets to these pools. So investors can add their crypto tokens like ETH or DAI to a liquidity pool, often they have to select a specific pool like a ETH-DAI trading pool. These pools then collect tokens from many providers and any user that wants to exchange one token for another toke in the pool can do a trade. A trade is a swap with the pool. In return to contributing their tokens for trading in the pool, the investors receive LP tokens representing their part of the pool. If a users wants his original tokens back, he can hand in his LP tokens and will get his original assets including any fees from trading back. To summarize, liquidity pools are pools full of tokens needed for trades in DEXes, so they are crucial for decentralized trading and don't need an order book.

\item \textbf{Flash Loan}:
Flash loans are a tool in decentralized finance that allows borrowers to obtain loans without the risk of defaulting on the debt, all within a \textbf{single atomic transaction}. This is only possible because we can execute multiple actions in one transaction. Flash loans typically have low interest rates, ranging from 0.3\% to 0.4\%. Additionally flash loans have rather small transaction fees f.eg a transaction fee for a flash loan can be in the order of 100 USD even if the loan amount can grow beyond 1B USD (source :: Quiz lecture 5) But what if the borrower loses all his money he had in a flash loan ? If certain conditions are not met within the transaction, it will not be executed, ensuring the safety of the actors and assets involved. Hence we can add assertions to our code and if the  assertions fail, the transaction will not be executed. Flash loans bring a lot of opportunities. They enable risk-free arbitrage and wash trading, but still they have some risks and uncertainties. They can fail if there is unexpected slippage or if assumptions about arbitrage transactions do not hold due to other transactions being executed before them (called- pre-running). They can also be used for malicious attacks such as the Sandwich Attacks we covered in the lecture. 
\begin{figure}[h]
    \centering
    \includegraphics[width=0.5\textwidth]{Bildschirmfoto 2024-04-07 um 17.46.08.png} 
    \caption{The Concept Of Flash Loans \scriptsize{source: lecture 6}}
    \label{fig:DoS-attack}
\end{figure}
\begin{figure}[h]
    \centering
    \includegraphics[width=0.5\textwidth]{Bildschirmfoto 2024-04-07 um 17.46.18.png} 
    \caption{Flash Loan Arbitrage \scriptsize{source: lecture 6}}
    \label{fig:DoS-attack}
\end{figure}
\item \textbf{LP Tokens}: So called liquidity provider tokens, they are assets in decentralized finance (DeFi) that represent ownership in liquidity pools. Liquidity pools are components of AMMs, where users contribute their assets to facilitate/allow trading. In return for providing liquidity, users receive LP tokens proportional to their contribution. If the want to withdraw their assets the LP token serve as proof of ownership and allow the investor to get his money back.% did nt understand them fully yet.
\end{itemize}

\\%idk what i should do here Additionally, it can be beneficial to perform two successive liquidations. This approach allows you to gradually adjust your position until it reaches a healthier state. By liquidating a portion of your assets in each step, you can avoid sudden and substantial losses while maintaining greater control over your portfolio's risk exposure. This strategy emphasizes the importance of proactive risk management in DeFi lending and borrowing protocols.\\\\optimal to do 2 successive liquidations -> can liquidate until position is almost healthy but not yet and then liquidate another one. -> why is this good ?

\section{Different Crypto-Coins}
The world of cryptocurrencies is very diverse, with thousands of different assets circulating in the DeFi market. In this chapter, we will explore some of the cryptocurrency coins we also encountered during the course. Each currency has its own unique characteristics and functionalities but also its downsides. Here are some information's taken from the lecture slides:\\
\begin{figure}[h]
    \centering
    \includegraphics[width=0.5\textwidth]{Bildschirmfoto 2024-04-07 um 17.45.27.png} 
    \caption{Ampelforth \scriptsize{source: lecture 6}}
    \label{fig:DoS-attack}
\end{figure}
\begin{figure}[h]
    \centering
    \includegraphics[width=0.5\textwidth]{Bildschirmfoto 2024-04-07 um 17.45.18.png} 
    \caption{Ampelforth \scriptsize{source: lecture 6}}
    \label{fig:DoS-attack}
\end{figure}

\section{Exam Prep Material}
\begin{itemize}
    \item \textbf{Further Terminology: } They published two glossaries on Moodle, one about Ethereum and Blockchain technology, the other covers basic finance terminology.
    \item \textbf{Example Midterm Questions}: They were provided by the lecturers during FS 24 to prepare for the midterm examination:\\\\
    1. (Multiple Choice) What is the main challenge that oracles aim to solve in blockchain systems?\\

   a) Providing access to real-world events and data (correct)\\

   b) Enabling API query possibilities within the blockchain\\

   c) Allowing the blockchain to browse the Internet via wget\\

   d) Reducing the cost of writing data into the blockchain\\\\
2. (Multiple Choice) What is the main advantage of using an Automated Market Maker (AMM) in a decentralized exchange?\\

   a) It eliminates the need for order book maintenance (correct)\\

   b) It ensures no impermanent loss for liquidity providers\\

   c) It guarantees low slippage for all trades\\

   d) It prevents users from being vulnerable to sandwich attacks\\\\



3. (Open-ended) Discuss the role of mixers like Tornado Cash in the context of blockchain privacy and censorship. Explain how mixers work, the challenges they face, and the potential implications of their use.\\
    
    %allow for censorship, place a transaction and then mix it with everyone else in the ananomity set, provide a tool to have more privacy during a transaction and therefore avoid censorhsip as we then can withdraw our coins to a diffrent address. So they push privacy and therefore laos allwo us to avoid censoroship. To take part in a mixer a user places a transaction, the amount is usually predfiend by the mixer f.eg like 1 ETH and the your transaction will be "routed through" the mixer nad at the end the user can withdraw his ETH to a new address. As normally a whole bunch of people place their transaction into the mixer, an externak observator cannot backtrack who did which transaction. A reason for normalizing the transaction amount if to avoid that it can be used as a data-side channel to identfiy transactions. Potential implications of the mixer use is that it is also a platofrm for attacker. malicious users can obviously make us of mixers. 
\end{itemize}
%potential topics to cover: maker dao websites, defi lama etc.
\end{document}