\documentclass{article}
\usepackage{graphicx} % Required for inserting images


\title{%
  Decentralized Finance \\
  \large Inoffical Course Script}
\author{Sarah Kuhn}
\date{April 2024}


\begin{document}
\maketitle
\thispagestyle{empty} % Remove page number from the first page

\newpage
\pagenumbering{arabic} % Start page numbering from 1
\setcounter{page}{1} % Set page counter to 1

\section{Finance Basics}
To begin we will first cover some fiance basics and underline why we need finical markets and what value they bring. Then we will discuss the key differences between CeFi and DeFi.
\subsection{Financial Markets: Underlying assumptions} 
Before we  talk about financial markets and their role in the economy, we first fix on what we base our analysis and reasoning about financial markets on. Obviously financial markets have a lot of real-life complexities and depend on decision from billions of individuals. Thus the assumption we make should provides a simplified framework that makes it easier for  to model complex financial systems. Economists often assume rational behavior and market efficiency, which then allows them to develop models and theories that help explain observed phenomena and make predictions seen in the markets.
\begin{itemize}
    \item \textbf{Rational Expectations}

Definition: Rational expectations describe that individuals make predictions about the future based on all available information and use these predictions to make decisions. These expectations are considered "rational" because they are consistent with the information available at the time.

In Financial Markets: Rational expectations imply that investors incorporate all available information, including past prices, economic data, and news, into their forecasts of future asset prices. This means that asset prices reflect the collective wisdom and expectations of investors, leading to efficient market outcomes. It also implies that actors won't make constant mistakes in predicting the future, which also means predicting the risks. 

    \item \textbf{Efficient Financial Market Theory}

    Definition: Efficient market theory suggests that financial markets quickly and accurately incorporate all available information into asset prices. So "asset prices represent the best possible estimates of the risk attached to them" (from lecture slide). Thus the risk from a financial market or just an investment can be inferred through mathematical analysis.

    \end{itemize}

\subsection{Role of Financial Markets} 
Financial markets play crucial role in our economy as they serve several purposes:
\begin{itemize}
    \item \textbf{Raising Capital}: Think of the financial market as a giant fundraising platform. It's where companies and governments go to ask investors for money. They can do this by selling parts of their company or borrowing money. This cash infusion helps them operate their businesses or kick start new projects.
    
    \item \textbf{Economic signaling}: We could also call this "setting fair prices": In the financial market, prices of assets like stocks and bonds are determined by how much people want to buy or sell them. This process helps investors know if they're paying a reasonable price for what they're investing in.
    \item \textbf{Ease of trading}: The financial market makes it convenient for people to buy and sell assets like stocks and bonds whenever they need to. This flexibility is crucial because it allows investors to adjust their portfolios swiftly in response to changing circumstances. We often say that a financial market provides liquidity to investors.
    \item \textbf{Risk management}: Within the financial market, there are tools available to help individuals protect their investments from unexpected events that could cause losses, like sudden market fluctuations or unexpected inflation. This is called " hedging against risk". Investors can protect themselves from adverse movements of factors that affect their value of assets. One can use
    instruments like options, futures or swaps to hedge against a potential decrease in a stock's value. In simple words it means that a investors puts money on the fact that the assets may loss value. If that scenario happens, the investor will get profit from it as he "predicted" it correctly. Another way of risk management is diversification, which refers to spreading investments across different assets and assets classes ( f.eg geographic or  industry) to reduce overall risk in a portfolio. This is a strategy which aims to ensure that losses in one investment can be balanced by gains in other assets.
     \item \textbf{Economic growth}: Financial markets take on a crucial role in promoting economic growth. They provide capital to different economic actors like companies and governments. This allows them to use the new liquidity to expand their operations and undertake new projects f.eg invest in infrastructure. This is can in turn lead to more job opportunities and in general contribute to a flourishing economic.
\end{itemize}

\subsection{Stakeholders in Financial Markets} 
In the previous section we looked at the role of financial markets in an economy. Now we will have a closer look at the "actors", in finance lingo called stakeholders, involved in financial markets. A stakeholder is any individual or a  group that can affect or be affected by the actions and outcomes of these financial markets. Stakeholders is an umbrella term for :

\begin{itemize}
\item \textbf{Investors}: Investors are individuals or organizations that supply funds to companies and governments in exchange for ownership(f.eg through buying a stock you are basically "owner" of a fraction of the company) or promises of future returns. They come in various forms, from individual investors managing their personal portfolios to institutional investors like pension funds, which pool money from multiple investors and then invest on their behalf. Investors typically acquire financial assets such as stocks, bonds, or also ETF, aiming to grow their wealth over time through positive returns on their assets. You can also think of it like the investors lend their money to companies with the belief that in the future the company will have a return because they used the money profitable. And then as an investors you will get parts of this big win because you were so nice and provided liquidity through lending your money. :)

\item \textbf{Issuers}: Issuers, as the name suggest, issue something . Issuers are entities like companies or  governments, that offer financial instruments to investors in order to raise capital for various purposes. Through offering various financial products like stocks or bonds, issuers have access to funds and thus can get more liquidity. In this way they can use it for funding their own projects, expanding operations, or repay existing debt. If you as an investor buy a stock for a company, we say the company is the isseurs of the stock.

\item \textbf{Intermediaries}: Intermediaries serve as "the man in the middle" between investors and issuers, facilitating the buying and selling of financial assets in the market. Imagine whenever you would like to invest, you had to go to all companies personally and talk about investing options, obviously this wouldn't be efficient and scalable. Thus we have intermediaries like  banks and stock exchanges, which provide platforms for trading and  information and execution services. They also play a crucial role in matching buyers with sellers and ensuring. Overall they are important for smooth transactions and thus enhance market efficiency and accessibility.

\item \textbf{Regulators}:Regulators, as the name suggest, are responsible for overseeing and regulating the financial markets to ensure their integrity, fairness, and stability. They establish and enforce rules aka regulations and standards on how market participants, such as investors, issuers, and intermediaries are allowed to interact with each other and the market. Regulators main task is monitoring the market activities and in that way maintain market confidence and also are those who have an overview of all the connections and interdependence's among financial institutions. In that way they play a big role in prevent systemic risks, where the whole finical market collapses like happened in the past.

\item \textbf{Market data providers}: Market data providers are organizations that gather, process information about financial markets to participants. They compile data on asset prices, trading volumes, market trends, and other relevant metrics, offering valuable insights for investors, issuers, and regulators. Mostly we make investment decision based on data offered by market data providers. Access to this timely and accurate market data enables all stakeholders to make informed decisions.
\end{itemize}

\subsection{Risk of Financial Markets} 
So we have seen what financial markets bring to the economy and that they are like a bustling marketplace where people buy and sell all sorts of financial products, like stocks and bonds. But with all the e potential for profit also comes a fair share of risk. In this section, we'll take a closer look at what financial risk means and how the rise of DeFi is promising to change the game.
\subsection{Market Vulnerability} 
Financial markets are exposed to a variety of different factors where each factor bring its own risk.
\begin{itemize}
    \item \textbf{Systemic risk}: If we participate in financial markets we are automatically exposed to system risk. These affect the whole system rather than just a single actor. Systemic risks arise from different sources and affect the whole system. Examples are political events or change of regulations.
    \item \textbf{Market risk}:  These refer to losses due to changes directly inside the market such as a change in exchange rate or sinking  interest rates. 
    
     \item \textbf{Credit risk}:  Credit risks refer the the potential loss when the borrowers creditworthiness changed and he might be in financial distress and thus not able to repay the credit.
    \item \textbf{Operational risk}:  This risk describes the potential of internal operations and intermediaries failing. Examples would be an error in the system or a fraud.
    \item \textbf{Cyber Security Risk}: This is a rather new and upcoming risk. As financial markets use more and more technology, cyber attacks on financial institutions increase and lead to loss of sensitive data, theft and other disruptions. 

\end{itemize}
\subsection{The DeFi promise } 
In the last section we saw that financial markets which have a centralized structure aka everything goes over the financial market, bring a lot of risk potentials. Especially 2008 after the economic crises the trust and believe in such centralized financial structures decreased. In November 2008 Satoshi Nakamotot released a software that started Bitcoin and advertised it as a "fully peer-to-peer system with no trusted third party". For some this was the Birthday for Defi. But what is different from DeFi to CeFi?
\subsubsection{From CeFi to DeFi}
\begin{itemize}
    \item \textbf{Inclusivity}:DeFi aims to provide financial services to individuals globally, also to those without access to banking services.
    \item \textbf{Innovation}: DeFi platforms are mostly open-source and allow developers to create and deploy financial applications without needing approval from centralized institutions.
    \item \textbf{Reduced Cost}: As in DeFi there are not intermediates but more often automatic processes f.eg through smart contracts, DeFi has potential to have lower costs than traditional finance services.
    \item \textbf{Transparency}: Transparency is a big keyword in DeFi. Technologies such as blockchain are transparent as they allow each user to see and trace every transaction. For some this transparent manner enhances trust for this financial system.
    \item \textbf{Financial Empowerment}: DeFi promises to empower the individuals by giving them more control and access over their assets. In DeFi the user often have a so called private key which allows them to directly engage in financial exchanges etc, without needing a traditional intermediaries.

\end{itemize}




\end{document}
